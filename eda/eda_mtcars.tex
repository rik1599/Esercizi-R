% Options for packages loaded elsewhere
\PassOptionsToPackage{unicode}{hyperref}
\PassOptionsToPackage{hyphens}{url}
%
\documentclass[
]{article}
\author{}
\date{\vspace{-2.5em}}

\usepackage{amsmath,amssymb}
\usepackage{lmodern}
\usepackage{iftex}
\ifPDFTeX
  \usepackage[T1]{fontenc}
  \usepackage[utf8]{inputenc}
  \usepackage{textcomp} % provide euro and other symbols
\else % if luatex or xetex
  \usepackage{unicode-math}
  \defaultfontfeatures{Scale=MatchLowercase}
  \defaultfontfeatures[\rmfamily]{Ligatures=TeX,Scale=1}
\fi
% Use upquote if available, for straight quotes in verbatim environments
\IfFileExists{upquote.sty}{\usepackage{upquote}}{}
\IfFileExists{microtype.sty}{% use microtype if available
  \usepackage[]{microtype}
  \UseMicrotypeSet[protrusion]{basicmath} % disable protrusion for tt fonts
}{}
\makeatletter
\@ifundefined{KOMAClassName}{% if non-KOMA class
  \IfFileExists{parskip.sty}{%
    \usepackage{parskip}
  }{% else
    \setlength{\parindent}{0pt}
    \setlength{\parskip}{6pt plus 2pt minus 1pt}}
}{% if KOMA class
  \KOMAoptions{parskip=half}}
\makeatother
\usepackage{xcolor}
\IfFileExists{xurl.sty}{\usepackage{xurl}}{} % add URL line breaks if available
\IfFileExists{bookmark.sty}{\usepackage{bookmark}}{\usepackage{hyperref}}
\hypersetup{
  hidelinks,
  pdfcreator={LaTeX via pandoc}}
\urlstyle{same} % disable monospaced font for URLs
\usepackage[margin=1in]{geometry}
\usepackage{color}
\usepackage{fancyvrb}
\newcommand{\VerbBar}{|}
\newcommand{\VERB}{\Verb[commandchars=\\\{\}]}
\DefineVerbatimEnvironment{Highlighting}{Verbatim}{commandchars=\\\{\}}
% Add ',fontsize=\small' for more characters per line
\usepackage{framed}
\definecolor{shadecolor}{RGB}{248,248,248}
\newenvironment{Shaded}{\begin{snugshade}}{\end{snugshade}}
\newcommand{\AlertTok}[1]{\textcolor[rgb]{0.94,0.16,0.16}{#1}}
\newcommand{\AnnotationTok}[1]{\textcolor[rgb]{0.56,0.35,0.01}{\textbf{\textit{#1}}}}
\newcommand{\AttributeTok}[1]{\textcolor[rgb]{0.77,0.63,0.00}{#1}}
\newcommand{\BaseNTok}[1]{\textcolor[rgb]{0.00,0.00,0.81}{#1}}
\newcommand{\BuiltInTok}[1]{#1}
\newcommand{\CharTok}[1]{\textcolor[rgb]{0.31,0.60,0.02}{#1}}
\newcommand{\CommentTok}[1]{\textcolor[rgb]{0.56,0.35,0.01}{\textit{#1}}}
\newcommand{\CommentVarTok}[1]{\textcolor[rgb]{0.56,0.35,0.01}{\textbf{\textit{#1}}}}
\newcommand{\ConstantTok}[1]{\textcolor[rgb]{0.00,0.00,0.00}{#1}}
\newcommand{\ControlFlowTok}[1]{\textcolor[rgb]{0.13,0.29,0.53}{\textbf{#1}}}
\newcommand{\DataTypeTok}[1]{\textcolor[rgb]{0.13,0.29,0.53}{#1}}
\newcommand{\DecValTok}[1]{\textcolor[rgb]{0.00,0.00,0.81}{#1}}
\newcommand{\DocumentationTok}[1]{\textcolor[rgb]{0.56,0.35,0.01}{\textbf{\textit{#1}}}}
\newcommand{\ErrorTok}[1]{\textcolor[rgb]{0.64,0.00,0.00}{\textbf{#1}}}
\newcommand{\ExtensionTok}[1]{#1}
\newcommand{\FloatTok}[1]{\textcolor[rgb]{0.00,0.00,0.81}{#1}}
\newcommand{\FunctionTok}[1]{\textcolor[rgb]{0.00,0.00,0.00}{#1}}
\newcommand{\ImportTok}[1]{#1}
\newcommand{\InformationTok}[1]{\textcolor[rgb]{0.56,0.35,0.01}{\textbf{\textit{#1}}}}
\newcommand{\KeywordTok}[1]{\textcolor[rgb]{0.13,0.29,0.53}{\textbf{#1}}}
\newcommand{\NormalTok}[1]{#1}
\newcommand{\OperatorTok}[1]{\textcolor[rgb]{0.81,0.36,0.00}{\textbf{#1}}}
\newcommand{\OtherTok}[1]{\textcolor[rgb]{0.56,0.35,0.01}{#1}}
\newcommand{\PreprocessorTok}[1]{\textcolor[rgb]{0.56,0.35,0.01}{\textit{#1}}}
\newcommand{\RegionMarkerTok}[1]{#1}
\newcommand{\SpecialCharTok}[1]{\textcolor[rgb]{0.00,0.00,0.00}{#1}}
\newcommand{\SpecialStringTok}[1]{\textcolor[rgb]{0.31,0.60,0.02}{#1}}
\newcommand{\StringTok}[1]{\textcolor[rgb]{0.31,0.60,0.02}{#1}}
\newcommand{\VariableTok}[1]{\textcolor[rgb]{0.00,0.00,0.00}{#1}}
\newcommand{\VerbatimStringTok}[1]{\textcolor[rgb]{0.31,0.60,0.02}{#1}}
\newcommand{\WarningTok}[1]{\textcolor[rgb]{0.56,0.35,0.01}{\textbf{\textit{#1}}}}
\usepackage{graphicx}
\makeatletter
\def\maxwidth{\ifdim\Gin@nat@width>\linewidth\linewidth\else\Gin@nat@width\fi}
\def\maxheight{\ifdim\Gin@nat@height>\textheight\textheight\else\Gin@nat@height\fi}
\makeatother
% Scale images if necessary, so that they will not overflow the page
% margins by default, and it is still possible to overwrite the defaults
% using explicit options in \includegraphics[width, height, ...]{}
\setkeys{Gin}{width=\maxwidth,height=\maxheight,keepaspectratio}
% Set default figure placement to htbp
\makeatletter
\def\fps@figure{htbp}
\makeatother
\setlength{\emergencystretch}{3em} % prevent overfull lines
\providecommand{\tightlist}{%
  \setlength{\itemsep}{0pt}\setlength{\parskip}{0pt}}
\setcounter{secnumdepth}{-\maxdimen} % remove section numbering
\ifLuaTeX
  \usepackage{selnolig}  % disable illegal ligatures
\fi

\begin{document}

Let us consider the dataframe mtcars, which comprises the fuel
consumption and 10 aspects of design and performance for 32 automobiles
(1970s models). The help file is given below

Describe how to perform a preliminary data analysis on this dataframe,
using suitable R commands.

\hypertarget{scopo-e-analisi-delle-variabili}{%
\subsubsection{Scopo e analisi delle
variabili}\label{scopo-e-analisi-delle-variabili}}

Si vuole creare un modello statistico di regressione per analizzare i
consumi medi di alcuni modelli di auto presenti nel dataframe
\texttt{mtcars}.

La variabile risposta è \texttt{mpg} che rappresenta il numero di miglia
(americane) percorse (in media) con un gallone di carburante.

Analizziamo le variabili

\begin{Shaded}
\begin{Highlighting}[]
\FunctionTok{str}\NormalTok{(mtcars)}
\end{Highlighting}
\end{Shaded}

\begin{verbatim}
## 'data.frame':    32 obs. of  11 variables:
##  $ mpg : num  21 21 22.8 21.4 18.7 18.1 14.3 24.4 22.8 19.2 ...
##  $ cyl : num  6 6 4 6 8 6 8 4 4 6 ...
##  $ disp: num  160 160 108 258 360 ...
##  $ hp  : num  110 110 93 110 175 105 245 62 95 123 ...
##  $ drat: num  3.9 3.9 3.85 3.08 3.15 2.76 3.21 3.69 3.92 3.92 ...
##  $ wt  : num  2.62 2.88 2.32 3.21 3.44 ...
##  $ qsec: num  16.5 17 18.6 19.4 17 ...
##  $ vs  : num  0 0 1 1 0 1 0 1 1 1 ...
##  $ am  : num  1 1 1 0 0 0 0 0 0 0 ...
##  $ gear: num  4 4 4 3 3 3 3 4 4 4 ...
##  $ carb: num  4 4 1 1 2 1 4 2 2 4 ...
\end{verbatim}

Ci sono alcune variabili da convertire

\begin{Shaded}
\begin{Highlighting}[]
\NormalTok{mtcars}\SpecialCharTok{$}\NormalTok{cyl }\OtherTok{\textless{}{-}} \FunctionTok{as.integer}\NormalTok{(mtcars}\SpecialCharTok{$}\NormalTok{cyl)}
\NormalTok{mtcars}\SpecialCharTok{$}\NormalTok{hp }\OtherTok{\textless{}{-}} \FunctionTok{as.integer}\NormalTok{(mtcars}\SpecialCharTok{$}\NormalTok{hp)}
\NormalTok{mtcars}\SpecialCharTok{$}\NormalTok{vs }\OtherTok{\textless{}{-}} \FunctionTok{factor}\NormalTok{(mtcars}\SpecialCharTok{$}\NormalTok{vs)}
\FunctionTok{levels}\NormalTok{(mtcars}\SpecialCharTok{$}\NormalTok{vs) }\OtherTok{\textless{}{-}} \FunctionTok{c}\NormalTok{(}\StringTok{"V{-}shaped"}\NormalTok{, }\StringTok{"straight"}\NormalTok{)}
\NormalTok{mtcars}\SpecialCharTok{$}\NormalTok{am }\OtherTok{\textless{}{-}} \FunctionTok{factor}\NormalTok{(mtcars}\SpecialCharTok{$}\NormalTok{am)}
\FunctionTok{levels}\NormalTok{(mtcars}\SpecialCharTok{$}\NormalTok{am) }\OtherTok{\textless{}{-}} \FunctionTok{c}\NormalTok{(}\StringTok{"automatic"}\NormalTok{, }\StringTok{"manual"}\NormalTok{)}
\NormalTok{mtcars}\SpecialCharTok{$}\NormalTok{gear }\OtherTok{\textless{}{-}} \FunctionTok{as.integer}\NormalTok{(mtcars}\SpecialCharTok{$}\NormalTok{gear)}
\NormalTok{mtcars}\SpecialCharTok{$}\NormalTok{carb }\OtherTok{\textless{}{-}} \FunctionTok{as.integer}\NormalTok{(mtcars}\SpecialCharTok{$}\NormalTok{carb)}
\FunctionTok{str}\NormalTok{(mtcars)}
\end{Highlighting}
\end{Shaded}

\begin{verbatim}
## 'data.frame':    32 obs. of  11 variables:
##  $ mpg : num  21 21 22.8 21.4 18.7 18.1 14.3 24.4 22.8 19.2 ...
##  $ cyl : int  6 6 4 6 8 6 8 4 4 6 ...
##  $ disp: num  160 160 108 258 360 ...
##  $ hp  : int  110 110 93 110 175 105 245 62 95 123 ...
##  $ drat: num  3.9 3.9 3.85 3.08 3.15 2.76 3.21 3.69 3.92 3.92 ...
##  $ wt  : num  2.62 2.88 2.32 3.21 3.44 ...
##  $ qsec: num  16.5 17 18.6 19.4 17 ...
##  $ vs  : Factor w/ 2 levels "V-shaped","straight": 1 1 2 2 1 2 1 2 2 2 ...
##  $ am  : Factor w/ 2 levels "automatic","manual": 2 2 2 1 1 1 1 1 1 1 ...
##  $ gear: int  4 4 4 3 3 3 3 4 4 4 ...
##  $ carb: int  4 4 1 1 2 1 4 2 2 4 ...
\end{verbatim}

\begin{Shaded}
\begin{Highlighting}[]
\FunctionTok{summary}\NormalTok{(mtcars)}
\end{Highlighting}
\end{Shaded}

\begin{verbatim}
##       mpg             cyl             disp             hp       
##  Min.   :10.40   Min.   :4.000   Min.   : 71.1   Min.   : 52.0  
##  1st Qu.:15.43   1st Qu.:4.000   1st Qu.:120.8   1st Qu.: 96.5  
##  Median :19.20   Median :6.000   Median :196.3   Median :123.0  
##  Mean   :20.09   Mean   :6.188   Mean   :230.7   Mean   :146.7  
##  3rd Qu.:22.80   3rd Qu.:8.000   3rd Qu.:326.0   3rd Qu.:180.0  
##  Max.   :33.90   Max.   :8.000   Max.   :472.0   Max.   :335.0  
##       drat             wt             qsec              vs             am    
##  Min.   :2.760   Min.   :1.513   Min.   :14.50   V-shaped:18   automatic:19  
##  1st Qu.:3.080   1st Qu.:2.581   1st Qu.:16.89   straight:14   manual   :13  
##  Median :3.695   Median :3.325   Median :17.71                               
##  Mean   :3.597   Mean   :3.217   Mean   :17.85                               
##  3rd Qu.:3.920   3rd Qu.:3.610   3rd Qu.:18.90                               
##  Max.   :4.930   Max.   :5.424   Max.   :22.90                               
##       gear            carb      
##  Min.   :3.000   Min.   :1.000  
##  1st Qu.:3.000   1st Qu.:2.000  
##  Median :4.000   Median :2.000  
##  Mean   :3.688   Mean   :2.812  
##  3rd Qu.:4.000   3rd Qu.:4.000  
##  Max.   :5.000   Max.   :8.000
\end{verbatim}

\hypertarget{analisi-della-variabile-risposta}{%
\subsubsection{Analisi della variabile
risposta}\label{analisi-della-variabile-risposta}}

\begin{Shaded}
\begin{Highlighting}[]
\FunctionTok{library}\NormalTok{(moments)}

\FunctionTok{hist}\NormalTok{(mtcars}\SpecialCharTok{$}\NormalTok{mpg, }\AttributeTok{probability =}\NormalTok{ T, }\AttributeTok{breaks =} \DecValTok{10}\NormalTok{)}
\FunctionTok{lines}\NormalTok{(}\FunctionTok{density}\NormalTok{(mtcars}\SpecialCharTok{$}\NormalTok{mpg), }\AttributeTok{lwd=}\DecValTok{3}\NormalTok{)}
\FunctionTok{lines}\NormalTok{(}
  \FunctionTok{seq}\NormalTok{(}\DecValTok{0}\NormalTok{, }\DecValTok{50}\NormalTok{, }\FloatTok{0.01}\NormalTok{), }
  \FunctionTok{dnorm}\NormalTok{(}\FunctionTok{seq}\NormalTok{(}\DecValTok{0}\NormalTok{, }\DecValTok{50}\NormalTok{, }\FloatTok{0.01}\NormalTok{), }\AttributeTok{mean =} \FunctionTok{mean}\NormalTok{(mtcars}\SpecialCharTok{$}\NormalTok{mpg), }\FunctionTok{sqrt}\NormalTok{(}\FunctionTok{var}\NormalTok{(mtcars}\SpecialCharTok{$}\NormalTok{mpg))),}
  \AttributeTok{col =} \StringTok{"cyan"}\NormalTok{)}
\FunctionTok{abline}\NormalTok{(}\AttributeTok{v=}\FunctionTok{mean}\NormalTok{(mtcars}\SpecialCharTok{$}\NormalTok{mpg), }\AttributeTok{col =} \StringTok{"red"}\NormalTok{)}
\FunctionTok{abline}\NormalTok{(}\AttributeTok{v=}\FunctionTok{median}\NormalTok{(mtcars}\SpecialCharTok{$}\NormalTok{mpg), }\AttributeTok{col =} \StringTok{"green"}\NormalTok{)}
\end{Highlighting}
\end{Shaded}

\includegraphics{eda_mtcars_files/figure-latex/unnamed-chunk-5-1.pdf}

\begin{Shaded}
\begin{Highlighting}[]
\FunctionTok{skewness}\NormalTok{(mtcars}\SpecialCharTok{$}\NormalTok{mpg)}
\end{Highlighting}
\end{Shaded}

\begin{verbatim}
## [1] 0.6404399
\end{verbatim}

\begin{Shaded}
\begin{Highlighting}[]
\FunctionTok{kurtosis}\NormalTok{(mtcars}\SpecialCharTok{$}\NormalTok{mpg)}
\end{Highlighting}
\end{Shaded}

\begin{verbatim}
## [1] 2.799467
\end{verbatim}

Il grafico risulta leggermente decentrato rispetto alla distribuzione
normale, proviamo una trasformata

\begin{Shaded}
\begin{Highlighting}[]
\NormalTok{logmpg }\OtherTok{\textless{}{-}} \FunctionTok{log}\NormalTok{(mtcars}\SpecialCharTok{$}\NormalTok{mpg)}
\FunctionTok{hist}\NormalTok{(logmpg, }\AttributeTok{probability =}\NormalTok{ T, }\AttributeTok{breaks =} \DecValTok{10}\NormalTok{)}
\FunctionTok{lines}\NormalTok{(}\FunctionTok{density}\NormalTok{(logmpg), }\AttributeTok{lwd=}\DecValTok{3}\NormalTok{)}
\FunctionTok{lines}\NormalTok{(}
  \FunctionTok{seq}\NormalTok{(}\DecValTok{0}\NormalTok{, }\DecValTok{50}\NormalTok{, }\FloatTok{0.01}\NormalTok{), }
  \FunctionTok{dnorm}\NormalTok{(}\FunctionTok{seq}\NormalTok{(}\DecValTok{0}\NormalTok{, }\DecValTok{50}\NormalTok{, }\FloatTok{0.01}\NormalTok{), }\AttributeTok{mean =} \FunctionTok{mean}\NormalTok{(logmpg), }\FunctionTok{sqrt}\NormalTok{(}\FunctionTok{var}\NormalTok{(logmpg))),}
  \AttributeTok{col =} \StringTok{"cyan"}\NormalTok{)}
\FunctionTok{abline}\NormalTok{(}\AttributeTok{v=}\FunctionTok{mean}\NormalTok{(logmpg), }\AttributeTok{col =} \StringTok{"red"}\NormalTok{)}
\FunctionTok{abline}\NormalTok{(}\AttributeTok{v=}\FunctionTok{median}\NormalTok{(logmpg), }\AttributeTok{col =} \StringTok{"green"}\NormalTok{)}
\end{Highlighting}
\end{Shaded}

\includegraphics{eda_mtcars_files/figure-latex/unnamed-chunk-6-1.pdf}

\begin{Shaded}
\begin{Highlighting}[]
\FunctionTok{skewness}\NormalTok{(logmpg)}
\end{Highlighting}
\end{Shaded}

\begin{verbatim}
## [1] -0.02241292
\end{verbatim}

\begin{Shaded}
\begin{Highlighting}[]
\FunctionTok{kurtosis}\NormalTok{(logmpg)}
\end{Highlighting}
\end{Shaded}

\begin{verbatim}
## [1] 2.64908
\end{verbatim}

La trasformazione logaritmica sembra migliorare l'andamento della
densità di mpg, visto che ne aumenta sensibilmente la simmetria a
discapito di una leggera diminuzione dell' indice di curtosi.

\begin{Shaded}
\begin{Highlighting}[]
\NormalTok{mtcars}\SpecialCharTok{$}\NormalTok{logmpg }\OtherTok{\textless{}{-}}\NormalTok{ logmpg}
\end{Highlighting}
\end{Shaded}

\hypertarget{le-altre-variabili}{%
\subsubsection{Le altre variabili}\label{le-altre-variabili}}

\hypertarget{istogrammi-e-barplot-delle-variabili}{%
\subsubsection{Istogrammi e barplot delle
variabili}\label{istogrammi-e-barplot-delle-variabili}}

\begin{Shaded}
\begin{Highlighting}[]
\CommentTok{\#par(mfrow=c(2, 5))}

\FunctionTok{barplot}\NormalTok{(}\FunctionTok{table}\NormalTok{(mtcars[, }\StringTok{"cyl"}\NormalTok{]), }\AttributeTok{xlab =} \StringTok{"Numero di cilindri"}\NormalTok{)}
\end{Highlighting}
\end{Shaded}

\includegraphics{eda_mtcars_files/figure-latex/unnamed-chunk-8-1.pdf}

\begin{Shaded}
\begin{Highlighting}[]
\FunctionTok{hist}\NormalTok{(mtcars}\SpecialCharTok{$}\NormalTok{disp, }\AttributeTok{breaks =} \DecValTok{10}\NormalTok{)}
\end{Highlighting}
\end{Shaded}

\includegraphics{eda_mtcars_files/figure-latex/unnamed-chunk-8-2.pdf}

\begin{Shaded}
\begin{Highlighting}[]
\FunctionTok{hist}\NormalTok{(mtcars}\SpecialCharTok{$}\NormalTok{hp, }\AttributeTok{breaks =} \DecValTok{10}\NormalTok{)}
\end{Highlighting}
\end{Shaded}

\includegraphics{eda_mtcars_files/figure-latex/unnamed-chunk-8-3.pdf}

\begin{Shaded}
\begin{Highlighting}[]
\FunctionTok{hist}\NormalTok{(mtcars}\SpecialCharTok{$}\NormalTok{drat, }\AttributeTok{breaks =} \DecValTok{10}\NormalTok{)}
\end{Highlighting}
\end{Shaded}

\includegraphics{eda_mtcars_files/figure-latex/unnamed-chunk-8-4.pdf}

\begin{Shaded}
\begin{Highlighting}[]
\FunctionTok{hist}\NormalTok{(mtcars}\SpecialCharTok{$}\NormalTok{wt, }\AttributeTok{breaks =} \DecValTok{10}\NormalTok{)}
\end{Highlighting}
\end{Shaded}

\includegraphics{eda_mtcars_files/figure-latex/unnamed-chunk-8-5.pdf}

\begin{Shaded}
\begin{Highlighting}[]
\FunctionTok{hist}\NormalTok{(mtcars}\SpecialCharTok{$}\NormalTok{qsec, }\AttributeTok{breaks =} \DecValTok{10}\NormalTok{)}
\end{Highlighting}
\end{Shaded}

\includegraphics{eda_mtcars_files/figure-latex/unnamed-chunk-8-6.pdf}

\begin{Shaded}
\begin{Highlighting}[]
\FunctionTok{barplot}\NormalTok{(}\FunctionTok{table}\NormalTok{(mtcars[, }\StringTok{"vs"}\NormalTok{]), }\AttributeTok{xlab =} \StringTok{"vs"}\NormalTok{)}
\end{Highlighting}
\end{Shaded}

\includegraphics{eda_mtcars_files/figure-latex/unnamed-chunk-8-7.pdf}

\begin{Shaded}
\begin{Highlighting}[]
\FunctionTok{barplot}\NormalTok{(}\FunctionTok{table}\NormalTok{(mtcars[, }\StringTok{"am"}\NormalTok{]), }\AttributeTok{xlab =} \StringTok{"am"}\NormalTok{)}
\end{Highlighting}
\end{Shaded}

\includegraphics{eda_mtcars_files/figure-latex/unnamed-chunk-8-8.pdf}

\begin{Shaded}
\begin{Highlighting}[]
\FunctionTok{barplot}\NormalTok{(}\FunctionTok{table}\NormalTok{(mtcars[, }\StringTok{"gear"}\NormalTok{]), }\AttributeTok{xlab =} \StringTok{"gear"}\NormalTok{)}
\end{Highlighting}
\end{Shaded}

\includegraphics{eda_mtcars_files/figure-latex/unnamed-chunk-8-9.pdf}

\begin{Shaded}
\begin{Highlighting}[]
\FunctionTok{barplot}\NormalTok{(}\FunctionTok{table}\NormalTok{(mtcars[, }\StringTok{"carb"}\NormalTok{]), }\AttributeTok{xlab =} \StringTok{"carb"}\NormalTok{)}
\end{Highlighting}
\end{Shaded}

\includegraphics{eda_mtcars_files/figure-latex/unnamed-chunk-8-10.pdf}

\begin{Shaded}
\begin{Highlighting}[]
\CommentTok{\#par(mfrow=c(1,1))}
\end{Highlighting}
\end{Shaded}

\hypertarget{relazioni-tra-le-variabili}{%
\subsubsection{Relazioni tra le
variabili}\label{relazioni-tra-le-variabili}}

\begin{Shaded}
\begin{Highlighting}[]
\ControlFlowTok{for}\NormalTok{ (i }\ControlFlowTok{in} \FunctionTok{c}\NormalTok{(}\DecValTok{2}\NormalTok{, }\DecValTok{8}\NormalTok{, }\DecValTok{9}\NormalTok{, }\DecValTok{10}\NormalTok{)) \{}
  \FunctionTok{boxplot}\NormalTok{(mtcars}\SpecialCharTok{$}\NormalTok{logmpg}\SpecialCharTok{\textasciitilde{}}\NormalTok{mtcars[, i], }\AttributeTok{xlab =} \FunctionTok{names}\NormalTok{(mtcars)[i])}
\NormalTok{\}}
\end{Highlighting}
\end{Shaded}

\includegraphics{eda_mtcars_files/figure-latex/unnamed-chunk-9-1.pdf}
\includegraphics{eda_mtcars_files/figure-latex/unnamed-chunk-9-2.pdf}
\includegraphics{eda_mtcars_files/figure-latex/unnamed-chunk-9-3.pdf}
\includegraphics{eda_mtcars_files/figure-latex/unnamed-chunk-9-4.pdf}

\begin{Shaded}
\begin{Highlighting}[]
\ControlFlowTok{for}\NormalTok{ (i }\ControlFlowTok{in} \FunctionTok{c}\NormalTok{(}\DecValTok{3}\NormalTok{, }\DecValTok{4}\NormalTok{, }\DecValTok{5}\NormalTok{, }\DecValTok{6}\NormalTok{, }\DecValTok{7}\NormalTok{, }\DecValTok{11}\NormalTok{)) \{}
  \FunctionTok{plot}\NormalTok{(mtcars}\SpecialCharTok{$}\NormalTok{logmpg}\SpecialCharTok{\textasciitilde{}}\NormalTok{mtcars[, i], }\AttributeTok{xlab =} \FunctionTok{names}\NormalTok{(mtcars)[i])}
  \FunctionTok{lines}\NormalTok{(}\FunctionTok{lowess}\NormalTok{(mtcars}\SpecialCharTok{$}\NormalTok{logmpg}\SpecialCharTok{\textasciitilde{}}\NormalTok{mtcars[, i]))}
\NormalTok{\}}
\end{Highlighting}
\end{Shaded}

\includegraphics{eda_mtcars_files/figure-latex/unnamed-chunk-9-5.pdf}
\includegraphics{eda_mtcars_files/figure-latex/unnamed-chunk-9-6.pdf}
\includegraphics{eda_mtcars_files/figure-latex/unnamed-chunk-9-7.pdf}
\includegraphics{eda_mtcars_files/figure-latex/unnamed-chunk-9-8.pdf}
\includegraphics{eda_mtcars_files/figure-latex/unnamed-chunk-9-9.pdf}
\includegraphics{eda_mtcars_files/figure-latex/unnamed-chunk-9-10.pdf}
Sembra esserci una forte correlazione lineare tra log(mpg) e cyl, vs, am
e wt. Anche per le variabili hp, disp e drat sembra esserci
correlazione, ma non di tipo lineare.

Verifichiamo le correlazioni con gli indici di Pearson e Spearman

\begin{Shaded}
\begin{Highlighting}[]
\FunctionTok{cor}\NormalTok{(mtcars}\SpecialCharTok{$}\NormalTok{logmpg, mtcars}\SpecialCharTok{$}\NormalTok{cyl, }\AttributeTok{method =} \StringTok{"pearson"}\NormalTok{)}
\end{Highlighting}
\end{Shaded}

\begin{verbatim}
## [1] -0.8546921
\end{verbatim}

\begin{Shaded}
\begin{Highlighting}[]
\FunctionTok{cor}\NormalTok{(mtcars}\SpecialCharTok{$}\NormalTok{logmpg, mtcars}\SpecialCharTok{$}\NormalTok{wt, }\AttributeTok{method =} \StringTok{"pearson"}\NormalTok{)}
\end{Highlighting}
\end{Shaded}

\begin{verbatim}
## [1] -0.8930611
\end{verbatim}

\begin{Shaded}
\begin{Highlighting}[]
\CommentTok{\# ha senso calcolare la correlazione su variabili categoriali?}
\CommentTok{\#cor(mtcars$logmpg, as.numeric(mtcars$vs), method = "pearson")}
\CommentTok{\#cor(mtcars$logmpg, as.numeric(mtcars$vs), method = "spearman")}
\CommentTok{\#cor(mtcars$logmpg, as.numeric(mtcars$am), method = "pearson")}
\CommentTok{\#cor(mtcars$logmpg, as.numeric(mtcars$am), method = "spearman")}

\FunctionTok{cor}\NormalTok{(mtcars}\SpecialCharTok{$}\NormalTok{logmpg, mtcars}\SpecialCharTok{$}\NormalTok{hp, }\AttributeTok{method =} \StringTok{"pearson"}\NormalTok{)}
\end{Highlighting}
\end{Shaded}

\begin{verbatim}
## [1] -0.7894727
\end{verbatim}

\begin{Shaded}
\begin{Highlighting}[]
\FunctionTok{cor}\NormalTok{(mtcars}\SpecialCharTok{$}\NormalTok{logmpg, mtcars}\SpecialCharTok{$}\NormalTok{hp, }\AttributeTok{method =} \StringTok{"spearman"}\NormalTok{)}
\end{Highlighting}
\end{Shaded}

\begin{verbatim}
## [1] -0.8946646
\end{verbatim}

\begin{Shaded}
\begin{Highlighting}[]
\FunctionTok{cor}\NormalTok{(mtcars}\SpecialCharTok{$}\NormalTok{logmpg, mtcars}\SpecialCharTok{$}\NormalTok{disp, }\AttributeTok{method =} \StringTok{"pearson"}\NormalTok{)}
\end{Highlighting}
\end{Shaded}

\begin{verbatim}
## [1] -0.8804044
\end{verbatim}

\begin{Shaded}
\begin{Highlighting}[]
\FunctionTok{cor}\NormalTok{(mtcars}\SpecialCharTok{$}\NormalTok{logmpg, mtcars}\SpecialCharTok{$}\NormalTok{disp, }\AttributeTok{method =} \StringTok{"spearman"}\NormalTok{)}
\end{Highlighting}
\end{Shaded}

\begin{verbatim}
## [1] -0.9088824
\end{verbatim}

\begin{Shaded}
\begin{Highlighting}[]
\FunctionTok{cor}\NormalTok{(mtcars}\SpecialCharTok{$}\NormalTok{logmpg, mtcars}\SpecialCharTok{$}\NormalTok{drat, }\AttributeTok{method =} \StringTok{"pearson"}\NormalTok{)}
\end{Highlighting}
\end{Shaded}

\begin{verbatim}
## [1] 0.6729062
\end{verbatim}

\begin{Shaded}
\begin{Highlighting}[]
\FunctionTok{cor}\NormalTok{(mtcars}\SpecialCharTok{$}\NormalTok{logmpg, mtcars}\SpecialCharTok{$}\NormalTok{drat, }\AttributeTok{method =} \StringTok{"spearman"}\NormalTok{)}
\end{Highlighting}
\end{Shaded}

\begin{verbatim}
## [1] 0.6514555
\end{verbatim}

Il valore degli indici conferma alcune impressioni, ma ne modifica
altre:

\begin{itemize}
\item
  la forte correlazione lineare (negativa) è confermata per cyl e wt,
\item
  il confronto tra l'indice di pearson e spearman per hp e disp
  suggerisce la presenza di correlazioni tra queste due variabili e
  logmpg, ma non di tipo lineare (probabilmente sono inversamente
  proporzionali),
\item
  per quanto riguarda drat, il confronto tra i due indici suggerisce una
  correlazione lineare in contrasto con l'ipotesi fatta in precedenza
\end{itemize}

Proviamo a trasformare hp e wt

\begin{Shaded}
\begin{Highlighting}[]
\FunctionTok{plot}\NormalTok{(mtcars}\SpecialCharTok{$}\NormalTok{logmpg}\SpecialCharTok{\textasciitilde{}}\FunctionTok{I}\NormalTok{(}\DecValTok{1}\SpecialCharTok{/}\NormalTok{mtcars}\SpecialCharTok{$}\NormalTok{hp))}
\FunctionTok{lines}\NormalTok{(}\FunctionTok{lowess}\NormalTok{(mtcars}\SpecialCharTok{$}\NormalTok{logmpg}\SpecialCharTok{\textasciitilde{}}\FunctionTok{I}\NormalTok{(}\DecValTok{1}\SpecialCharTok{/}\NormalTok{mtcars}\SpecialCharTok{$}\NormalTok{hp)))}
\end{Highlighting}
\end{Shaded}

\includegraphics{eda_mtcars_files/figure-latex/unnamed-chunk-11-1.pdf}

\begin{Shaded}
\begin{Highlighting}[]
\FunctionTok{cor}\NormalTok{(mtcars}\SpecialCharTok{$}\NormalTok{logmpg, }\DecValTok{1}\SpecialCharTok{/}\NormalTok{mtcars}\SpecialCharTok{$}\NormalTok{hp, }\AttributeTok{method =} \StringTok{"pearson"}\NormalTok{)}
\end{Highlighting}
\end{Shaded}

\begin{verbatim}
## [1] 0.8366077
\end{verbatim}

\begin{Shaded}
\begin{Highlighting}[]
\FunctionTok{plot}\NormalTok{(mtcars}\SpecialCharTok{$}\NormalTok{logmpg}\SpecialCharTok{\textasciitilde{}}\FunctionTok{I}\NormalTok{(}\DecValTok{1}\SpecialCharTok{/}\NormalTok{mtcars}\SpecialCharTok{$}\NormalTok{wt))}
\FunctionTok{lines}\NormalTok{(}\FunctionTok{lowess}\NormalTok{(mtcars}\SpecialCharTok{$}\NormalTok{logmpg}\SpecialCharTok{\textasciitilde{}}\FunctionTok{I}\NormalTok{(}\DecValTok{1}\SpecialCharTok{/}\NormalTok{mtcars}\SpecialCharTok{$}\NormalTok{wt)))}
\end{Highlighting}
\end{Shaded}

\includegraphics{eda_mtcars_files/figure-latex/unnamed-chunk-11-2.pdf}

\begin{Shaded}
\begin{Highlighting}[]
\FunctionTok{cor}\NormalTok{(mtcars}\SpecialCharTok{$}\NormalTok{logmpg, }\DecValTok{1}\SpecialCharTok{/}\NormalTok{mtcars}\SpecialCharTok{$}\NormalTok{wt, }\AttributeTok{method =} \StringTok{"pearson"}\NormalTok{)}
\end{Highlighting}
\end{Shaded}

\begin{verbatim}
## [1] 0.8642357
\end{verbatim}

I grafici e il valore degli indici di pearson confermano la bontà delle
due trasformate.

\end{document}
